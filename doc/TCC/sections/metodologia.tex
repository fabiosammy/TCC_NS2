\section{Metodologia dos testes} 
Neste trabalho, o objetivo das simula\c{c}\~oes \'e permitir uma an\'alise do comportamento de tr\^es protocolos de roteamento em redes \textit{ad hoc} (AODV, DSDV e OLSR), sob a influ\^encia de um cen\'ario que retrata uma aplica\c{c}\~ao militar, revelando os problemas decorrentes da utiliza\c{c}\~ao deste tipo de rede em um cen\'ario com essas caracter\'isticas, e buscando as melhores condi\c{c}\~oes para contornar estes problemas. Al\'em disso, por meio dsa simula\c{c}\~oes realizadas podemos avaliar o impacto que a mobilidade em grupo, a configura\c{c}\~ao de rede hier\'arquica e o movimento dos n\'os em uma dire\c{c}\~ao pr\'e-determinada podem causar no roteamento dos dados.

\subsection{Ambiente de simula\c{c}\~ ao}
Para o trabalho apresentado, os testes s\~ao baseados em simula\c{c}\~oes geradas pelo resultado de um projeto colaborativo entre a Universidade da Calif\'ornia do Sul e o laborat\'orio Xerox PARC, o simulador NS-2.

O NS-2 \'e um simulador de eventos discreto, oferecendo suporte \`a simula\c{c}\~ao de um grande n\'umero de topologias de redem duferentes cen\'arios baseados nos protocolos TCP e UDP, diversos escalonadores e pol\'iticas de fila, caracteriza\c{c}\~ao de tr\'afego com diversas distribui\c{c}\~oes estat\'isticas dentre outras finalidades.

\subsection{Cen\'arios militares}
\subsubsection{Requisitos b\'asicos}
\subsubsection{Movimenta\c{c}\~ao}



