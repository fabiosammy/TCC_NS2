\section{Metodologia dos testes}\label{metodologia}
Neste trabalho, o objetivo das simula\c{c}\~oes \'e permitir uma an\'alise do comportamento de tr\^es protocolos de roteamento em redes \textit{ad hoc} (AODV, DSDV e OLSR), sob a influ\^encia de dois cen\'arios que retratam uma aplica\c{c}\~ao militar, revelando os problemas decorrentes da utiliza\c{c}\~ao deste tipo de rede em um cen\'ario com essas caracter\'isticas, e buscando as melhores condi\c{c}\~oes para contornar estes problemas. 
Al\'em disso, por meio das simula\c{c}\~oes realizadas podemos avaliar o impacto que a mobilidade em grupo, a configura\c{c}\~ao de rede hier\'arquica e o movimento dos n\'os em uma dire\c{c}\~ao pr\'e-determinada podem causar no roteamento dos dados.

\subsection{Ambiente de simula\c{c}\~ ao}
Para o trabalho apresentado, os testes s\~ao baseados em um simulador desenvolvido em um projeto colaborativo entre a Universidade da Calif\'ornia do Sul e o laborat\'orio Xerox PARC, o simulador NS-2 \cite{FallVaradhan}.

O NS-2 \'e um simulador de eventos discreto, oferecendo suporte \`a simula\c{c}\~ao de um grande n\'umero de topologias de redes diferentes cen\'arios baseados nos protocolos TCP e UDP, diversos escalonadores e pol\'iticas de fila, caracteriza\c{c}\~ao de tr\'afego com diversas distribui\c{c}\~oes estat\'isticas dentre outras finalidades.

\subsection{Cen\'arios militares}
Segundo \cite{pereira}, em um cen\'ario de opera\c{c}\~oes militares \'e indispens\'avel o uso de tecnologias de comunica\c{c}\~ao sem fio. 
Principalmente, das tecnologias que tenham capacidade de atuar de forma aut\^onoma e proporcionar qualidade de conex\~ao juntamente com o grau de dinamicidade na topologia da rede.
A qualidade da rede de comuni\c{c}\~ao utilizada em uma opera\c{c}\~ao militar pode acarretar o sucesso ou fracasso da miss\~ao.

\subsubsection{Requisitos b\'asicos}\label{requisitos}
Segundo \cite{salles}, v\'arios requisitos s\~ao necess\'arios para que uma rede \textit{ad hoc} possa funcionar efetivamente em um cen\'ario real militar, onde o autor enumera v\'arios princ\'ipios de emprego das comunica\c{c}\~oes militares no Ex\'ercito Brasileiro, que s\~ao:
\begin{description}
	\item[Tempo integral:] Operar 24 horas por dia, todos os dias.
	\item[Rapidez:] Estabelecer contato em tempo \'util para surtir os efeitos desejados.
	\item[Amplitude de desdobramento:] Estar operacional em todo o teatro de opera\c{c}\~oes.
	\item[Integra\c{c}\~ao:] Operar junto com os sistemas dos escal\~oes superior e inferior.
	\item[Flexibilidade:] Adequar-se rapidamente \`as mudan\c{c}as das opera\c{c}\~oes t\'aticas e das oraganiza\c{c}\~oes militares.
	\item[Apoio em profundidade:] Apoio ao escal\~ao superior (mais recuado) para com os escal\~oes subordinados (mais avan\c{c}ado).
	\item[Continuidade:] Retomar as comunica\c{c}\~oes e mant\^e-las a qualquer custo, mesmo que o escal\~ao considerado n\~ao seja o respons\'avel.
	\item[Confiabilidade:] Estar sempre dispon\'ivel, estabelecendo caminhos alternativos para a transmiss\~ao das mensagens.
	\item[Emprego centralizado:] Concentrar meios em centros e eixos de comunica\c{c}\~oes permitindo melhor aproveitamento dos mesmos.
	\item[Apoio cerrado:] Encurtar as dist\^ancias sempre que poss\'ivel para facilitar as comunica\c{c}\~oes.
	\item[Seguran\c{c}a:] Impedir ou pelo menos dificultar a obten\c{c}\~ao da informa\c{c}\~ao pelo inimigo.
	\item[Prioridade:] Estabelecer comunica\c{c}\~ao e transmitir mensagens de acordo com a prioridade preestabelecida.
\end{description}

Esses mesmos princ\'ipios foram mapeados pelos autores \cite{salles} em outros cinco outros termos utilizados comumente na concep\c{c}\~ao de redes de comunica\c{c}\~oes. Abaixo a tabela \ref{tabExer} demonstra o mapeamento.
\begin{table}[H]
	\centering
	\caption{Mapeamento em princ\'ipios gerais dos princ\'ipios de emprego das comunica\c{c}\~oes militares no Ex\'ercito Brasileiro \cite{salles}}
	\begin{tabular}{ | l | l | }
		\hline
		\textbf{Princ\'ipios gerais} & \textbf{Princ\'ipios de emprego das comunica\c{c}\~oes militares} \\ \hline
		Escalabilidade & Amplitude de desdobramento, Integra\c{c}\~ao. \\ \hline
		Desempenho & Tempo integral, Rapidez, Confiabilidade, Continuidade, Prioridade \\ \hline
		Seguran\c{c}a & Seguran\c{c}a \\ \hline
		Gerenciabilidade & Apoio em profundidade, Emprego centralizado, Apoio cerrado \\ \hline
		Usabilidade & Flexibilidade \\ \hline
	\end{tabular}
	\label{tabExer}
\end{table}

Baseando-se em tais informa\c{c}\~oes descritas por \cite{salles} e estudos realizados por \cite{pereira}, definiu-se o mapeamento de "Desempenho" realizado por \cite{salles} como base para criar as m\'etricas de desempenho para comparar os protocolos de roteamento.

\subsection{M\'etricas de desempenho}
Com base nos estudo de \cite{pereira} e \cite{schimidt} e informa\c{c}\~oes obtidas por \cite{salles} descritos na Se\c{c}\~ao \ref{requisitos}, foram definidas m\'etricas para analisar o comportamento de cada protocolo.

\begin{itemize}
	\item \textbf{Taxa de entrega de pacotes:} Raz\~ao entre o n\'umero de pacotes entregues para o destino final e o n\'umero de pacotes gerados pela aplica\c{c}\~ao na fonte.
	\item \textbf{Atraso m\'edio fim a fim dos pacotes de dados:} Inclui todos os poss\'iveis atrasos causados pela lat\^encia da descoberta de rotas, propaga\c{c}\~ao, atrasos devido a retransmiss\~oes da camada MAC e tempos de transfer\^encia.
	\item \textbf{N\'umero de pacotes de roteamento:} \'E medido a quantidade total de pacotes de roteamento, representada pelos pacotes de descoberta e manuten\c{c}\~ao das rotas enviados pela origem ou encaminhados pelos n\'os intermedi\'arios. No protocolo por demanda (AODV), esses pacotes s\~ao representados pelos pacotes RREQ, RREP e RERR. No DSDV e OLSR, esses pacotes s\~ao representados pelas tabelas de roteamento que s\~ao trocadas periodicamente.
	\item \textbf{N\'umero de \textit{bytes} de roteamento:} \'E medido a quantidade total de \textit{bytes} em cada pacote de, incluindo a quantidade de \textit{bytes} de cabe\c{c}alho em pacotes de dados, que corresponde, normalmente, ao roteamento na fonte.
\end{itemize}
Segundo \cite{pereira}, o tr\'afego referente ao roteamento deve ser o menor poss\'ivel quando comparado ao tr\'afego de dados, pois para se enviar pacotes de roteamento gasta-se energia dos n\'os e consome-se banda, que s\~ao recursos escassos em redes sem fio. 
Para as redes militares, a taxa de entrega e o atraso s\~ao as m\'etricas mais importantes.

\subsection{Tr\'afego de dados}\label{trafegoDados}
Para obter uma compara\c{c}\~ao justa entre todos os protocolos, \'e necess\'ario selecionar um agente que n\~ao crie condi\c{c}\~oes de desigualdade, como mecanismos pr\'oprios de controle de congestionamento.
O simulador utilizado, NS-2, oferece a possibilidade de gerar conex\~oes de tr\'afego TCP(\textit{Transport Control Protocol}) ou CBR(\textit{Constant Bit Rate}).
O documento \cite{rfc793} define que o TCP possui um mecanismo pr\'oprio de controle de congestionamento, e tamb\'em os pacotes de reconhecimento (ACK) do agente disputam o canal, podendo causar colis\~oes e degradar o desempenho.
Com base nisso, o CBR(\textit{Constant Bit Rate}) foi selecionado, o qual usa agente UDP(\textit{User Datagram Protocol}), que n\~ao possui um controle de congestionamento pr\'oprio \cite{rfc768}.

\subsection{Leitura dos resultados}

\begin{verbatim}
M 0.10000 0 (60.00, 10.00, 0.00), (60.00, 460.00), 1.50
M 0.10000 1 (200.00, 50.00, 0.00), (200.00, 150.00), 1.50
M 0.10000 2 (340.00, 10.00, 0.00), (340.00, 460.00), 1.50
M 0.10000 3 (200.00, 300.00, 0.00), (200.00, 300.00), 1.50
s -t 0.100000000 -Hs 2 -Hd -2 -Ni 2 -Nx 340.00 -Ny 10.00 -Nz 0.00 
 -Ne -1.000000 -Nl AGT -Nw --- -Ma 0 -Md 0 -Ms 0 -Mt 0 -Is 2.0 
 -Id 0.0 -It cbr -Il 512 -If 0 -Ii 0 -Iv 32 -Pn cbr -Pi 0 -Pf 0 -Po 0 
r -t 0.100000000 -Hs 2 -Hd -2 -Ni 2 -Nx 340.00 -Ny 10.00 -Nz 0.00 
 -Ne -1.000000 -Nl RTR -Nw --- -Ma 0 -Md 0 -Ms 0 -Mt 0 -Is 2.0 
 -Id 0.0 -It cbr -Il 512 -If 0 -Ii 0 -Iv 32 -Pn cbr -Pi 0 -Pf 0 -Po 0 
s -t 0.100000000 -Hs 2 -Hd -2 -Ni 2 -Nx 340.00 -Ny 10.00 -Nz 0.00 
 -Ne -1.000000 -Nl RTR -Nw --- -Ma 0 -Md 0 -Ms 0 -Mt 0 -Is 2.255 
 -Id -1.255 -It AODV -Il 48 -If 0 -Ii 0 -Iv 30 -P aodv -Pt 0x2 -Ph 1 
 -Pb 1 -Pd 0 -Pds 0 -Ps 2 -Pss 4 -Pc REQUEST 
s -t 0.100115000 -Hs 2 -Hd -2 -Ni 2 -Nx 340.00 -Ny 10.00 -Nz 0.00 
 -Ne -1.000000 -Nl MAC -Nw --- -Ma 0 -Md ffffffff -Ms 2 -Mt 800 
 -Is 2.255 -Id -1.255 -It AODV -Il 106 -If 0 -Ii 0 -Iv 30 -P aodv 
 -Pt 0x2 -Ph 1 -Pb 1 -Pd 0 -Pds 0 -Ps 2 -Pss 4 -Pc REQUEST 
r -t 0.100963485 -Hs 1 -Hd -2 -Ni 1 -Nx 200.00 -Ny 50.00 -Nz 0.00 
 -Ne -1.000000 -Nl MAC -Nw --- -Ma 0 -Md ffffffff -Ms 2 -Mt 800 
 -Is 2.255 -Id -1.255 -It AODV -Il 48 -If 0 -Ii 0 -Iv 30 -P aodv 
 -Pt 0x2 -Ph 1 -Pb 1 -Pd 0 -Pds 0 -Ps 2 -Pss 4 -Pc REQUEST 
r -t 0.100988485 -Hs 1 -Hd -2 -Ni 1 -Nx 200.00 -Ny 50.00 -Nz 0.00 
 -Ne -1.000000 -Nl RTR -Nw --- -Ma 0 -Md ffffffff -Ms 2 -Mt 800 
 -Is 2.255 -Id -1.255 -It AODV -Il 48 -If 0 -Ii 0 -Iv 30 -P aodv 
 -Pt 0x2 -Ph 1 -Pb 1 -Pd 0 -Pds 0 -Ps 2 -Pss 4 -Pc REQUEST
\end{verbatim}

