\section{Conclus\~ao e trabalhos futuros}

As simula\c{c}\~oes apresentadas neste trabalho confirmam o fato de que cada protocolo apresenta vantagens e desvantagens, dependendo dos condi\c{c}\~oes que lhe s\~ao impostas. 
Nos cen\'arios com prop\'ositos militares, que utilizam redes \textit{ad hoc}com configura\c{c}\~ao hier\'arquica, a entrega dos pacotes de forma eficiente e r\'apida \'e de extrema relev\^ancia. 
O protocolo "ideal" atenderia a todas as restri\c{c}\~oes impostas pelas necessidades de comunica\c{c}\~ao em cen\'arios tipicamente militares, mas o que se p\^ode concluir \'e a apartir dos resultados alcan\c{c}ados \'e que cada um dos protocolos avaliados mostrou sua diferen\c{c}a em determinada m\'etrica ou condi\c{c}\~ao.

Entre os protocolos analizados, podemos ver que o DSDV e o AODV tiveram comportamento opostos nos 2 diferentes experimentos realizados, e que o OLSR manteve uma boa estabilidade entre ambos os experimentos.

Para futuras pesquisas, \'e sugerido que seja combinado algumas caracter\'isticas de protocolos, prevalecendo das situa\c{c}\~oes onde em que eles apresentam maiores vantagens que se adaptem aos cen\'arios propostos. Al\'em disso, deve-se avaliar o impacto das implementa\c{c}\~oes relativas \`a seguran\c{c}a nesse tipo de rede, como mencionados em \cite{salles}.


