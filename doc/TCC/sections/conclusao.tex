\section{Conclus\~ao e trabalhos futuros}\label{conclusao}

As simula\c{c}\~oes apresentadas neste trabalho confirmam o fato de que cada protocolo apresenta vantagens e desvantagens, dependendo das condi\c{c}\~oes que lhe s\~ao impostas, como no caso, o n\'umero de n\'os na rede. 
Nos cen\'arios com prop\'ositos militares, que utilizam redes \textit{ad hoc} com configura\c{c}\~ao hier\'arquica, a entrega dos pacotes de forma eficiente e r\'apida \'e de extrema relev\^ancia. 
O protocolo "ideal"\ atenderia a todas as restri\c{c}\~oes impostas pelas necessidades de comunica\c{c}\~ao em cen\'arios tipicamente militares, mas o que se p\^ode concluir, a partir dos resultados alcan\c{c}ados, \'e que cada um dos protocolos avaliados mostrou sua diferen\c{c}a em determinada m\'etrica ou condi\c{c}\~ao.

Entre os protocolos analisados, pode ser observado que o DSDV e o AODV tiveram comportamentos opostos nos dois experimentos realizados, e que o OLSR manteve uma boa estabilidade entre ambos os experimentos.

Para futuras pesquisas, sugere-se que seja combinada algumas caracter\'isticas de protocolos, prevalecendo das situa\c{c}\~oes em que eles apresentam maiores vantagens nos cen\'arios propostos.
Bem como, sugere-se que sejam testados outros protocolos e mudan\c{c}as de cen\'arios.
Aos interessados em continuar o desenvolvimento desse trabalho, ou ent\~ao reproduzir os experimentos, \'e poss\'ivel obter todos os cen\'arios, \textit{script} de extra\c{c}\~ao dos dados, e a vers\~ao deste artigo em latex, acessando o link dispon\'ivel em \cite{fabiosammy}.
