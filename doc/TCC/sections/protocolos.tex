\section{Protocolos de roteamento em redes ad hoc}
Em redes m\'oveis ad hoc, uma rota entre dois n\'os pode ser formada por v\'arios saltos atrav\'es de um ou mais n\'os na rede. 

Os roteadores em uma rede ad hoc trocam informa\c{c}\~oes de roteamento uns com os outros com a finalidade de tomar conhecimento das disponibilidades de rotas e da topologia da rede.\cite{pereira}

\subsection{Ad hoc on demand distance vector - AODV}
Reativo, inicia as atividades de roteamento de acordo com a demanda.


\subsection{Destination sequenced distance vector - DSDV}
Pr\'o-ativo, mant\'em rotas para todos os n\'os da rede, independente do uso dessa rota.


\subsection{Optimized link state routing - OLSR}
Pr\'o-ativo, marca\c{c}\~ao especial dos vizinhos.

