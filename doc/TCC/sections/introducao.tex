\section{Introdu\c{c}\~ao} 
Atualmente a maioria da popula\c{c}\~ao possui um dispositivo port\'atil com disponibilidade de conex\~ao em redes sem-fio e capaz de se comunicar com outros dispositivos. 
Com a populariza\c{c}\~ao destes dispositivos, \'e muito comum existirem in\'umeras redes sem-fio dispon\'iveis para conex\~ao abertamente, ou a possibilidade de se criar uma nova conforme necessidade.

As redes sem-fio s\~ao chamadas de \textit{wireless}, elas provem comunica\c{c}\~ao, seja com uma pequena ou grande rede de comunica\c{c}\~ao. 
Nesse tipo de comunica\c{c}\~ao existem duas divis\~oes, as redes infra-estruturadas e as redes \textit{ad hoc}. 
As redes \textit{ad hoc} s\~ao consideradas MANET's(textit{Mobile ad hoc network}), pois n\~ao \'e necess\'ario uma estrutura fixa para que a comunica\c{c}\~ao funcione.
No tipo de rede infra-estruturada, os n\'os m\'oveis, mesmo pr\'oximos um do outro, est\~ao impossibilitados de estabelecer comunica\c{c}\~ao direta entre si \cite{pereira}.

As redes-infraestruturadas s\~ao compostas por uma Esta\c{c}\~ao de Suporte a Mobilidade (ESM), que gera o sinal, oferecendo acesso aos \textit{hosts}. 
Os \textit{hosts} s\'o geram conex\~oes, aplica\c{c}\~oes dos usu\'arios, e quem gera as rotas, o roteador, \'e o ESM. 
Por\'em as MANETs n\~ao t\^em o ESM, mas tem uma dinamicidade grande em sua forma de comunica\c{c}\~ao, pois a cada novo n\'o na rede, ele dever\'a atuar como um roteador e como um \textit{host}, executando aplica\c{c}\~oes dos usu\'arios. 
Para as redes \textit{ad hoc}, existem poucos protocolos oficiais de roteamento da rede, pois \'e um grande desafio o roteamento nesse tipo de rede pela forma din\^ amica de como a topologia da rede \'e desenvolvida.

\begin{quote}
"As MANETs proporcionam a exist\^encia de estruturas de comunica\c{c}\~ao em ambientes, por exemplo, com muitos obst\'aculos \`a cria\c{c}\~ao de uma estrutura fixa. 
Um cen\'ario de opera\c{c}\~ao militar pode ser visto como uma situa\c{c}\~ao em que as MANETs s\~ao requiridas para proporcionar a comunica\c{c}\~ao em um ambiente hostil e geograficamente acidentado, n\~ao possibilitando a exist\^encia de uma estrutura fixa"\cite{schimidt}.
\end{quote}

Os autores \cite{pepe} comentam sobre a dinamicidade das redes \textit{ad hoc} e a facilidade em criar uma rede nesse tipo, um cen\'ario militar \'e uma \'otima aplica\c{c}\~ao desse tipo de rede, pois tal cen\'ario \'e um ambiente hostil e geograficamente acidentado, n\~ao possibilitando a exist\^encia de uma estrutura fixa \cite{schimidt}.

Segundo \cite{salles}, a necessidade de uma comunica\c{c}\~ao r\'apida para a equipe militar \'e essencial para obter um sucesso em sua operac\~ao, pois qualquer atraso na comunicac\~ao pode influenciar em um final catastr\'ofico na execuc\~ao de um passo na operac\~ao. 
Os diversos protocolos existentes das redes \textit{ad hoc} diferenciam-se em n\'umero de pacotes de requisic\~oes a transitar na rede, modo de atualizac\~ao das rotas, sistema de armazenamento de rotas, e outros. o que influencia na performance final da rede.

O objetivo deste artigo \'e apresentar um estudo de diferentes protocolos de roteamento em redes ad hoc em cen\'arios militares, atrav\'es de m\'etricas de performance baseados em estudos realizados por \cite{pereira} e \cite{salles}.
